\documentclass{book}
\usepackage{geometry}
\geometry{a5paper, left=20mm, right=20mm, top=10mm}
\pagenumbering{gobble}
\begin{document}
\chapter*{Snowball's Chance by \textit{John Reed}}
\LARGE{\textbf{THE OLD PIGS WERE DYING.\\
		FIRST,	IT WAS Dominicus a secondary functionary who had given
	over his life to that rather crucial task of
	interpreting and graphing
statistical data. He wore black-rimmed glasses and liked to sing
opera as he sat at his desk where, one drizzly afternoon,
he collapsed into a plate of Camembert.
By unanimous proclamation, he
was named Animal Hero, First Class.
The next pig to die was
Napoleon himself. The great Berkshire boar.
The father of all animals.
Saviour of equality, liberty and freedom. He had died in
a manner fit to his station passing in his
sleep, between sheets of Egyptian
cotton (with an extremely high thread-count).\\\\
In commemorative tribute, a twelve-foot statue of Napoleon was
erected outside the barnhouse, at the former site of Old Major's
skull, for those who remembered Old Major, the pig who had
started it all, and those days those early, early days.\\\\
The statue was bronze; Napoleon wore his black coat and his leather
leggings. Standing on his hind-legs, he puffed his pipe
and looked to the horizon. Behind the statue, painted in white letters
on the tar wall of the barn, was the single Commandment-\textit{Most animals
are equalish}. To the left of the Commandment, the verses of
\textit{Founding Father Napoleon} were painted in the same white letters
. The poem, dedicated to the fallen leader, was authorised by Minimus,
who was known to be a pig with a poetic soul-\\\\}}
\LARGE{\textbf{\textit{Napoleon taught us how to read.\\
		Napoleon gives us grass and feed.\\
		Napoleon shows us bread can rise,\\
		With a swill o' swell guidance from the swine.\\
		The pigs are a species of splendorous knowing,\\
		But also of helping, and also of
		showing\\
		That the animals of the Manor Farm\\
		Are the tippest-toppest animals anywhere!\\
	    Gosh-darn!\\
		So let's give a honk and a quack and a squeak!\\
		An oink and a moo and a whinny and a peep!\\
		Let's doodle-doo, let's snort, and let's baaa!\\
		Let's give a bark and a hoot and a caaw!\\
		Don't hold it back! You squeal and you neigh!\\
		Napoleon, Napoleon,
		you're king am\\
		-ongst the hay!\\
		Napoleon, Napoleon, we know you'll lead the way!\\
Napoleon, Napoleon, guide us everyday!\\\\}}
\LARGE{\textbf{To further observe the accomplishments of the great
	Leader, the portrait of Napoleon, which surmounted the poem
	was refreshed. Six pigeons, with a retouch of colour, gave
	dimension to the white profile-under the tutelage of the pigs,
	the birds had acquired the skill of rendering.\\
	In the year that followed, several mor\\
	-e of the
	old-time swiners	cast off their mortal coil.
	One would drown in the bathtub (through no
	fault of his own)
	when he found himself unable to get out.
	Another would fall victim to a swollen liver-downing
	his last mug of whiskey,
	he quietly moved on to the next life.
	Yet another would die
	of a patient torturer called
	cancer-fortunately, as he had long taken
	up Napoleon's habit of enjoying a good pipe
	several times an hour,
	he was offered much consolation in his
	final months. All, heroes of the
	rebellion, were declared Animal Heroes,
	First Class.\\\\
	The younger pigs filled their places
	well-enough, it seemed,
	though they were a reserved generation more
	aloof, and perhaps, more lenient.
	They were led by their elder, Squealer, who for years
	had been Napoleon's chief counselor. He was a
	pig who could wag his
    tail and tongue quite persuasively-so
	much so that in the end, he may have
	convinced even himself that he was a pig
	of the populace. Though he had been saying for
	years that rations were increasing, for
    the first time that anyone could remember\\
	(aside from the pigs, who were always firm
	in their conviction that\\
	things were always
	getting better) it seemed possible that the
	ration-bag was a little rounder-and noticeably
	so. When Squealer died, he himself had grown
	so fat that he was blinded by his own face.
	The cause of death, it was pronounced,
	was over-work.\\
	In another first (or at least the first that
	anyone could remember), this pronouncement
	by the pigs was openly derided. At the
	posthumous awards presentation (Animal Hero,
	First Class), there were even a few stealthy
	hecklers-hooters and honkers. Squealer wasn't
	so terrible, after all-but surely, a pig who in
	his last days was pushed around in
    a wheelbarrow, as he could not even sustain his
	girth on four legs, was no pig
    who had, as it was claimed, died of "a lifetime
	of exertion". It would have taken old Squealer
	himself to explain that a pig buried in a piano
	case wasn't funny.\\
	The last of the old pigs to take control was
	Minimus. He, like the others before him, was
	considered one of the original heroes of the
	rebellion. (And yet his ascent was cause for
	much surprise, as aside from compose a few poems,
	nobody could accurately pinpoint what he had done.)
	Though robust, Minimus was quite advanced in
	years-and to address concerns that the next
	succession might be turbulent, Pinkeye, the
	most powerful, and incidentally, well-liked
	pig of the younger generation, was selected
	to fill the newly created position, Next Leader.
    So Pinkeye kissed ducklings and lambs, as
	Minimus went about managing the farm. A silent
	Leader, Minimus was a mystery to be feared and
	respected. The dogs were loyal to his service,
	as were the other pigs, just as it had always
	been.\\
	And yet there was a new calm
	unprecedented a
	calm bespeaking, perhaps, a better future, or
	perhaps, the darkness of days to come.\\
	It was one night-an average sort of normal
	May night-that there was an
	extra
extraordinary
    disturbance
	in
	the stalls. The moon low on
	the horizon, a figure had appeared at the gates.
	It was a strange figure-unfamiliar in his dark
	suit with pleated pants and a wide-lapel. The animal
    (was it an animal?) walked on two feet, wore shoes
	and a brimmed hat, and carried a briefcase. A few
	steps behind him, a goat was similarly accoutered.
    (Was it a goat? Yes, it was a goat. Surely, a
	most sophisticated goat.)\\
	The dog in attendance at the outer gate barked
	ferociously at the pair-though not many of
	the barn animals paid him much mind, as the
	dogs at the outer gate were particularly high
	-strung beasts, known to be incited to woof
	by causes of innocuous as moon shadows and
	silverfish. One of the cows, no doubt bolstered
	by the anonymity of night, belted out her exasperation
	at having been, once again, so rudely awoken-\\
	"Shaat-up!"\\
	In actuality, however, the scene that took place
	at the outer gate was not nearly so common as
	the cow imagin\\
	-ed for although the cause of the
	shepherd's excitement was a stranger and not of a
	silverfish, after what seemed scarcely more than
	a few well-chosen words, the guard dog, having
	dropped to his forelegs, was backing away on his belly.
	Mouth closed, eyes wide, he lowered his head and
	tucked his tail under his haunches-as the briefcased
	pair, cutting elegant if foreboding silhouettes
	against the indigo sky, breached the outer gate
with no more discussion.\\\\
From her perch in the
hayloft, Norma the cat,
who had been watching
the moon through the
chinks in the barn,
was the single
animal to witness
the brief exchange. Norma, like most cats,
was more interested
in being a cat than
a member of the Manor
Farm. Yet she was
always an
extremely personable
creature; always playful.
And excepting those times
she was lazing around
in the shade while
other animals were huffing
in the sun, she was widely
appreciated.\\\\
``Sssssssss!'' she hissed at
the broken windowpane, her back arched,
her claws extended, her hair on end.\\\\
This, as would be expected,
immediately woke the rats,
who endeavored to keep
themselves well attuned to the cat.}}




\end{document}
